\AtAppendix{\counterwithin{equation}{section}}
%********************************
\chapter{Anhang}
\label{ch:appendix}
%********************************
\section{Beweis des Optimismus der euklidischen Distanz als Heuristik}
\label{sec:beweis_heuristik}
%********************************
\textbf{Beweis: Die Euklidische Distanz ist immer geringer als die Oktile Distanz und daher eine optimistische Heuristik.} \\
Zu zeigen:
\begin{equation}
    \label{eq:beweis_heuristik_bedingung}
    d_\text{eukl} \leq d_\text{okt}\)
\end{equation}

Wir betrachten die Distanzen für beliebige Punkte $(x_1, y_1)$ und $(x_2, y_2)$.\\
Betrachten wir den Fall, dass  $|x_2 - x_1| = |y_2 - y_1|$.
Dann gilt:
\begin{align*}
    d_\text{eukl} &= \sqrt{2} \cdot |x_2 - x_1|\\
    d_\text{okt} &= |x_2 - x_1| + (\sqrt{2}-1) \cdot |x_2 - x_1| = |x_2 - x_1| \cdot \sqrt{2}
\end{align*}
In diesem Fall sind die Distanzen identisch, was der Bedingung aus Gl. \ref{eq:beweis_heuristik_bedingung} entspricht.

Betrachten wir nun den Fall $|x_2 - x_1| \neq |y_2 - y_1|$.
Ohne Beschränkung der Allgemeinheit nehmen wir an $|x_2 - x_1| > |y_2 - y_1|$.
Dann gilt:
\begin{align*}
    d_\text{eukl} &= \sqrt{(x_2 - x_1)^2 + (y_2 - y_1)^2}\\
    d_\text{okt} &= |x_2 - x_1| + (\sqrt{2}-1) \cdot |y_2 - y_1|
\end{align*}

Nun muss die folgende Ungleichung bewiesen werden:
\begin{equation*}
    \sqrt{(x_2 - x_1)^2 + (y_2 - y_1)^2} \leq |x_2 - x_1| + |y_2 - y_1|
\end{equation*}

Da beide Seiten der Ungleichung nicht-negativ sind, ist das Quadrieren beider Seiten gültig und bewahrt die Richtung der Ungleichung.
Löst man den binomialen Term auf, ergibt sich:
\begin{equation*}
(x_2 - x_1)^2 + (y_2 - y_1)^2 \leq \left(|x_2 - x_1|\right)^2 + 2|x_2 - x_1||y_2 - y_1| + \left(|y_2 - y_1|\right)^2
\end{equation*}

Aufgrund von $|a|^2 = a^2$ für jede Zahl $a\in\mathbb{R}$ wird die Ungleichung zu:
\begin{equation*}
(x_2 - x_1)^2 + (y_2 - y_1)^2 \leq (x_2 - x_1)^2 + 2|x_2 - x_1||y_2 - y_1| + (y_2 - y_1)^2
\end{equation*}
Vereinfacht ergibt sich:
\begin{equation*}
    0 \leq 2|x_2 - x_1||y_2 - y_1|
\end{equation*}
Der Ausdruck $2|x_2 - x_1||y_2 - y_1|$ ist immer positiv, da es sich um das Produkt von Beträgen handelt, die selbst positiv sind.

Daraus folgt, dass für alle möglichen Paare $(x_1, y_1)$ und $(x_2, y_2)$:
\begin{equation*}
    d_\text{Euklidisch} \leq d_\text{Oktile}
\end{equation*}
Damit ist gezeigt, dass die Euklidische Distanz eine optimistische Heuristik darstellt.