%***********************************************************************************************%
%===============================================================================================%
% 																								%
%										Ausarbeitung zum Thema  								%
% 																								%
%               "Implementierung einer GUI-basierten Pathfinding-Algorithmen Benchmarking       %
%                            Software im Modul 'Effizient Programmieren'" 					    %
% 																								%
%									 von Felix Renzikowski, 2024							    %
% 																								%
%===============================================================================================%
%***********************************************************************************************%


% ============= Klassendefinition ============= %

\documentclass[
	%draft, 		% Entwurfsstadium
	final, 			% fertiges Dokument
	paper=a4,
	pagesize=auto,
	fontsize=12pt,
	ngerman,
	openright,
	twoside,		% Aktivieren, falls doppelseitiger Druck
	numbers=noendperiod,
	bibliography=totoc,
	listof=totoc
]{scrreprt}

% ============= Verwendete Pakete ============= %

% Dokumentinformationen
\usepackage{hyperref}
\hypersetup{
	pdftitle={Implementierung einer GUI-basierten Pathfinding-Algorithmen Benchmarking Software im Modul 'Effizient Programmieren'},
	pdfsubject={Abschlussausarbeitung},
	pdfauthor={Felix Renzikowski},
	pdfkeywords={C++, CMake, Qt5, Effizient programmieren},
	pdfborder={0 0 0},
	colorlinks=true,
	breaklinks=true,
	linktocpage=true,
	citecolor=black,
	linkcolor=black,	
	menucolor=black,	
	urlcolor=black
}
\urlstyle{same}

% Allgemeine Pakete
\usepackage[T1]{fontenc}
\usepackage[utf8]{inputenc}	% Sonderzeichen
\usepackage{graphicx,color}
\usepackage[ngerman]{babel}
\usepackage{geometry}
\usepackage{amssymb}
\usepackage{amsmath}
\usepackage{acronym}
\usepackage[labelfont=bf]{caption}
\usepackage{tocloft}
\usepackage{enumitem}
\usepackage{float}
\usepackage[headsepline,automark,markcase=ignoreuppercase]{scrlayer-scrpage}
\usepackage{pdfpages}
\usepackage{tikz}
\usepackage{pgfplots}
\pgfkeys{/pgf/number format/.cd, use comma, 1000 sep={.}}
\usepgfplotslibrary{groupplots}
\usepackage{pgfgantt}
\usepackage{pdflscape}
\usepackage{afterpage}
\pgfplotsset{compat=newest}
\usepackage{grffile}
\usepackage{subcaption}
\usepackage{lmodern}
\usepackage{siunitx}
\usepackage{listings}
\usepackage[export]{adjustbox}
\sisetup{output-decimal-marker = {,}}
\usepackage{mathtools}
\usepackage[mode=buildnew, subpreambles=true]{standalone}
\usepackage{wrapfig}
\usepackage{anyfontsize}
\usepackage[ngerman=ngerman-x-latest]{hyphsubst}
\raggedbottom

\lstset{ % Global setup
	basicstyle=\fontsize{6}{7}\selectfont\ttfamily,
	numbers=left,
	numberstyle=\tiny,
	keywordstyle=\color{blue},
	commentstyle=\color{green},
	stringstyle=\color{red},
	frame=tb,
	columns=fullflexible,
	breaklines=true,
	postbreak=\mbox{\textcolor{red}{$\hookrightarrow$}\space}
}

\makeatletter
\renewcommand{\fnum@figure}{Abb. \thefigure}
\makeatother
\renewcommand{\lstlistingname}{Quelltext}

% Einstellungen der Kopf- und Fußzeilen
\ohead{\headmark}									% Oben Rechts etwas angeben
\chead{}											% Keine Ausgabe oben in der Mitte
\cfoot[]{}											% Keine Seitenzahl in der Mitte
\ofoot[\pagemark]{\pagemark}						% Seitenzahlen rechts
%\automark{section}									% Sektionen in Kopfzeile (Einseitiger Modus)
\renewcommand*{\headfont}{\normalfont}				% Kein kursiver Text in Kopfzeile
\pagestyle{scrheadings}

% Absätze und Abstände
\renewcommand{\baselinestretch}{1}\normalsize		% Zeilenabstand
\setlength{\parindent}{0pt}							% Kein Einrücken nach Absatz
\geometry{inner=31.5mm,outer=31.5mm,bottom=3cm}		% Abstand zu unterem Seitenrand
\textwidth147mm

%%%%%%%%%%%%%%%%%%%%%%%%%%%%%%%%%%%%%%%%%%%%%%%%%%%%%%%%%%%%%%
% ============= 	 Beginn des Dokuments  	   ============= %
%%%%%%%%%%%%%%%%%%%%%%%%%%%%%%%%%%%%%%%%%%%%%%%%%%%%%%%%%%%%%%

\begin{document}
\pagenumbering{Roman}							% Römische Seitenzahlen bis zum Hauptteil
% ============= Titelseite =========== %
\newpage
\begin{center}
	\begin{Large}
	\textbf{Implementierung einer GUI-basierten Pathfinding-Algorithmen Benchmarking Software im Modul 'Effizient Programmieren'}\\
	\vspace{1cm}
	\textbf{Implementation of a GUI-based pathfinding algorithm benchmarking software in the module 'Effizient Programmieren'}
\end{Large}

\vspace{4cm}

Ausarbeitung\\
von\\
Felix Renzikowski

\vspace{4cm}

Durchgef\"uhrt am\\
\vspace{0.5cm}
\textbf{INSTITUT F\"UR AERODYNAMIK UND GASDYNAMIK}\\
\textbf{UNIVERSIT\"AT STUTTGART}\\

\vspace{2cm}

Dozent:\\
PD Dr. rer. nat. Manuel Keßler


\vspace{2cm}

Stuttgart, im September 2024
\end{center}
\addtocounter{page}{-1}

% ============= Vorspann ============= %
\pagestyle{scrheadings}
\renewcommand*{\chapterheadstartvskip}{\vspace*{2.3\baselineskip}}	% Abstand einstellen
\begingroup
\pdfbookmark[0]{Inhaltsverzeichnis}{toc}							% Inhaltsverzeichnis im pdf als Bookmark anzeigen
\renewcommand{\baselinestretch}{1.2}\normalsize						% Zeilenabstand
\let\clearpage\relax
\tableofcontents													% Inhaltsverzeichnis
\endgroup
\renewcommand{\baselinestretch}{1.25}\normalsize					% Zeilenabstand
\renewcommand*{\chapterheadstartvskip}{\vspace*{2.3\baselineskip}}	% Abstand einstellen

% ============= Symbolverzeichnis ============= %
\chapter*{Nomenklatur}
\addcontentsline{toc}{chapter}{Nomenklatur}
\renewcommand{\baselinestretch}{1.1}\normalsize						% Zeilenabstand
\begin{longtable}{p{0.2\textwidth} p{0.8\textwidth}}
    \hline
    Symbol & Bedeutung\\
    \hline
    \endhead
    $d_{okt}$               & oktile Distanz\\
    $d_{eukl}$              & euklidische Distanz\\
    $G(n)$                  & G-Kosten, tatsächliche Distanz\\
    $H(n)$                  & H-Kosten, heuristische Distanz\\
    $F(n)$                  & F-Kosten, $F(n)=G(n)+H(n)$\\
    $d_{eukl}$              & euklidische Distanz\\
    $\mathcal{O}$           & Landau-Symbol, Maß für Anzahl der Elementarschritte einer algorithmischen Operation\\
    \caption*{}
\end{longtable}

\begin{table}[H]
    \begin{tabular}{ll}
        \hline
        Abkürzung\phantom{123456}& Bedeutung \phantom{1234567891234567891234567891234567899}\\
        \hline
    \end{tabular}
\end{table}
\begin{addmargin}[0.2cm]{0cm}
    \vspace{-0.6cm}
    \begin{acronym}[MMAE~~~~~~~~~~~~]
        \setlength{\itemsep}{-\parsep}
        \acro{JPS}{Jump-Point-Search}
        \acro{IDE}{Integrated Development Environment}
        \acro{CI}{Continuous Integration}
        \acro{CD}{Continuous Delivery}
        \acro{GUI}{Graphical User Interface}
        \acro{CLI}{Command Line Interface}
        \acro{UML}{Unified Modeling Language}
    \end{acronym}
\end{addmargin}
\renewcommand{\baselinestretch}{1.25}\normalsize					% Zeilenabstand
\markleft{Nomenklatur}

\clearpage
\setlength{\cftfigindent}{0em}
\renewcommand{\baselinestretch}{1.1}\normalsize
\pdfbookmark{Abbildungsverzeichnis}{lof}
\addcontentsline{toc}{chapter}{Abbildungsverzeichnis}
\listoffigures 														% Abbildungsverzeichnis
\markleft{Abbildungsverzeichnis}

\clearpage
\pdfbookmark[0]{Tabellenverzeichnis}{tbv}
\addcontentsline{toc}{chapter}{Tabellenverzeichnis}
\setlength{\cfttabindent}{0em}
\listoftables														% Tabellenverzeichnis
\renewcommand{\baselinestretch}{1}\normalsize

\renewcommand{\lstlistlistingname}{Quelltextverzeichnis}
\begingroup
\pdfbookmark{Quelltextverzeichnis}{qtv}
\setlength{\cfttabindent}{0em}
\let\cleardoublepage\clearpage
\lstlistoflistings  % Code-Verzeichnis
\endgroup
\renewcommand{\baselinestretch}{1}\normalsize

% ============= Hauptteil ============= %
\cleardoubleoddpage

\pagenumbering{arabic}  											% Arabische Zahlen im Hauptteil bis zum Anhang

%********************************
\chapter{Einleitung}
\label{ch:einleitung}
%********************************

%********************************
\section{Zielsetzung}
\label{sec:motivation}
%********************************
Im Rahmen des Moduls `Effizient Programmieren' im Masterstudiengang Luft- und Raumfahrttechnik an der Universität Stuttgart
wurde ein GUI-basiertes Programm zur Lösung von 2D Grids mithilfe diverser Pathfinding-Algorithmen in C++ entwickelt.
Ziel des Projekts war es, die effiziente Gestaltung von Codeprojekten und Entwicklungsprozessen zu erlernen und anzuwenden.
Der Schwerpunkt lag hierbei sowohl auf der algorithmischen Optimierung als auch auf der Implementierung moderner Softwareentwicklungsmethoden und -werkzeuge.

%********************************
\section{Projektbeschreibung}
\label{sec:beschreibung}
%********************************
Das entwickelte Programm ermöglicht die Visualisierung und Lösung von 2D Grids unter Verwendung verschiedener Pathfinding-Algorithmen wie A*, Dijkstra und \ac{JPS}.
Über ein \ac{GUI} können Anwender Start- und Zielpunkte definieren.
Die Algorithmen berechnen den optimalen Pfad und visualisieren diesen in Echtzeit.
Außerdem können im Multi-Run Modus des Programms laufzeitoptimierte Benchmarks der implementierten Algorithmen mit verschiedenen Grid-Konfigurationen durchgeführt werden.
%********************************
\subsection*{Verwendete Tools und Technologien}
%********************************
Zur Umsetzung des Projekts kamen unter anderem folgende Tools und Technologien zum Einsatz:
\begin{itemize}
\item \textbf{\textit{CMake}:} Einsatz als plattformunabhängiges Build-System zur effizienten Verwaltung von Abhängigkeiten und Compiler-Optionen.
\item \textbf{\ac{CI}:} Einrichtung einer CI-Pipeline zur kontinuierlichen Überprüfung des Codes und Ausführung automatisierter Tests.
\item \textbf{\ac{IDE}:} Verwendung von \textit{CLion} (JetBrains).
\item \textbf{Versionskontrolle:} Verwendung von \textit{Git} und \textit{GitHub} zur Versionskontrolle und zum nahtlosen Arbeiten auf verschiedenen Geräten.
\end{itemize}

%********************************
\subsection*{Optimierungen}
%********************************
Im Entwicklungsprozess wurden mehrere Optimierungstechniken angewendet, um die Effizienz des Programms zu steigern:
\begin{itemize}
\item \textbf{Algorithmische Optimierungen:} Verbesserung des A*- und \ac{JPS}-Algorith\-mus durch eine geeignete Wahl der Heuristikfunktion und den Einsatz effizienter Datenstrukturen.
\item \textbf{Speicheroptimierungen:} Reduktion des Speicherverbrauchs durch den Einsatz von smarten Zeigern und das Vermeiden unnötiger Kopien.
\item \textbf{Parallelisierung:} Parallelisierung von Berechnungsteilen und \ac{GUI}-Event-Loop zur Aufrechterhaltung der \ac{GUI}-Interaktion.
\end{itemize}

\subsection*{Lerneffekte}
Das Projekt ermöglichte wertvolle Erkenntnisse in verschiedenen Bereichen der effizienten Softwareentwicklung:
\begin{itemize}
\item Vertiefung moderner C++-Techniken und Best Practices.
\item Erfahrungen in der Einrichtung und Nutzung von CI-Pipelines zur Verbesserung der Codequalität und -stabilität.
\item Verbesserung der Fähigkeiten im Schreiben und Strukturieren von Tests.
\item Effiziente Nutzung von Entwicklungswerkzeugen und -umgebungen zur Steigerung der Produktivität.
\end{itemize}
Zusammenfassend hat das Projekt nicht nur zur Verbesserung der technischen Fähigkeiten beigetragen, sondern auch die Kompetenz in der Anwendung effizienter Entwicklungsprozesse und -methoden gestärkt.
In den folgenden Kapiteln werden die einzelnen Komponenten, Optimierungstechniken und Lerneffekte detailliert erläutert.





%\include{02_grundlagen}

%********************************
\chapter{Programmaufbau und -struktur}
\label{ch:aufbau}
%********************************

%********************************
\section{Aufbau des Backends}
\label{sec:aufbau_backend}
%********************************
In diesem Abschnitt soll ein Überblick über den Aufbau und die Strukturierung des Backends gegeben werden.
`Backend' bezeichnet hierbei den ausführenden Teil des Programms, also Objekte und Strukturen die zur Berechnung/Lösung des Grids notwendig sind und keinen direkten User-Kontakt haben.

%********************************
\subsection{Überblick}
\label{subsec:aufbau_backend_ueberblick}
%********************************
Das Backend ist in zwei Namespaces unterteilt: `GridGenerator' und `Pathfinder'.
Ersterer beinhaltet jegliche Funktionalität die zur Erstellung der Grid-Objekte, zum Platzieren der Hindernisse und zur Positionierung des Start- und Endpunkts notwendig sind.
Der Namespace `Pathfinder' gruppiert Komponenten die zur Berechnung und Lösung ebendieser Objekte notwendig sind, sowie Funktionen zur Überwachung dieses Prozesses.
Im Folgenden soll zunächst auf die Erstellung der Grid-Objekte und sodann auf den Namespace `Pathfinder' eingegangen werden.
%********************************
\subsubsection{Namespace: GridGenerator}
\label{subsubsec:aufbau_backend_ueberblick_gridgen}
%********************************
Dieser Namespace setzt sich zusammen aus den Klassen \texttt{Cell}, \texttt{Grid} und \texttt{ObstacleGenerator}, sowie einigen Subklassen, unterstützenden Datenstrukturen und Enumerationen.
\begin{itemize}
    \item \textttt{Cell} \\
    \begin{figure}[H]
        \vspace{-0.5cm}
        \centering
        \includestandalone[width=0.7\textwidth]{assets/uml/cell_class_uml}
        \caption{UML-Klassendiagramm der Klasse \texttt{Cell}}
        \label{fig:uml_cell}
        %\vspace{-0.5cm}
    \end{figure}
    Die \texttt{Cell}-Klasse hält Informationen und Funktionalität einer spezifischen Gridzelle.
    Jede Zelle hält ihre Kosten (\texttt{CellCost}) und einen Zustand (\texttt{CellState}: Hindernis, begehbar, etc.).
    Außerdem enthält die \texttt{Cell}-Klasse mehrere Helferfunktionen für die Interaktion mit den Gitterzellen.

    \item \textttt{Grid} \\
    \begin{figure}[H]
        \vspace{-0.5cm}
        \centering
        \includestandalone[width=0.7\textwidth]{assets/uml/grid_class_uml}
        \caption{UML-Klassendiagramm der Klasse \texttt{Cell}}
        \label{fig:uml_grid}
        %\vspace{-0.5cm}
    \end{figure}
    Die \texttt{Grid}-Klasse hält Informationen und Funktionalität die Interaktionen mit dem Grid ermöglichen.
    Das tatsächliche Grid ist als \textttt{std::vector<std::vector<Cell>>} aufgebaut, der Zugrif findet über den \texttt{()}
    Operator und die Struktur \texttt{GridCoordinate} statt.
    Bei Instantiierung eines \texttt{Grid}-Objekts wird eine Referenz zu einem \texttt{GridGenerator}-Objekt (ähnlich eines Decorator-Pattern)
    welchem die Verantwortung für die Hindernisgenerierung und das Festsetzen des Start- und Endpunkts übergeben wird.


\end{itemize}

%\include{04_ergebnisse}

%\include{09_zusammenfassung_und_ausblick}

% ============= Literaturverzeichnis ============= %
\renewcommand{\baselinestretch}{1.2}\normalsize
\clearpage						% Zeilenabstand
\bibliographystyle{plaindin}
%\addcontentsline{toc}{chapter}{Literaturverzeichnis}
\bibliography{bibluS}
\renewcommand{\baselinestretch}{1}\normalsize						% Zeilenabstand

% =============Anhang ============= %
%\appendix
%\AtAppendix{\counterwithin{equation}{section}}
%********************************
\chapter{Anhang}
\label{ch:appendix}
%********************************
\section{Beweis des Optimismus der euklidischen Distanz als Heuristik}
\label{sec:beweis_heuristik}
%********************************
\textbf{Beweis: Die Euklidische Distanz ist immer geringer als die Oktile Distanz und daher eine optimistische Heuristik.} \\
Zu zeigen:
\begin{equation}
    \label{eq:beweis_heuristik_bedingung}
    d_\text{eukl} \leq d_\text{okt}\)
\end{equation}

Wir betrachten die Distanzen für beliebige Punkte $(x_1, y_1)$ und $(x_2, y_2)$.\\
Betrachten wir den Fall, dass  $|x_2 - x_1| = |y_2 - y_1|$.
Dann gilt:
\begin{align*}
    d_\text{eukl} &= \sqrt{2} \cdot |x_2 - x_1|\\
    d_\text{okt} &= |x_2 - x_1| + (\sqrt{2}-1) \cdot |x_2 - x_1| = |x_2 - x_1| \cdot \sqrt{2}
\end{align*}
In diesem Fall sind die Distanzen identisch, was der Bedingung aus Gl. \ref{eq:beweis_heuristik_bedingung} entspricht.

Betrachten wir nun den Fall $|x_2 - x_1| \neq |y_2 - y_1|$.
Ohne Beschränkung der Allgemeinheit nehmen wir an $|x_2 - x_1| > |y_2 - y_1|$.
Dann gilt:
\begin{align*}
    d_\text{eukl} &= \sqrt{(x_2 - x_1)^2 + (y_2 - y_1)^2}\\
    d_\text{okt} &= |x_2 - x_1| + (\sqrt{2}-1) \cdot |y_2 - y_1|
\end{align*}

Nun muss die folgende Ungleichung bewiesen werden:
\begin{equation*}
    \sqrt{(x_2 - x_1)^2 + (y_2 - y_1)^2} \leq |x_2 - x_1| + |y_2 - y_1|
\end{equation*}

Da beide Seiten der Ungleichung nicht-negativ sind, ist das Quadrieren beider Seiten gültig und bewahrt die Richtung der Ungleichung.
Löst man den binomialen Term auf, ergibt sich:
\begin{equation*}
(x_2 - x_1)^2 + (y_2 - y_1)^2 \leq \left(|x_2 - x_1|\right)^2 + 2|x_2 - x_1||y_2 - y_1| + \left(|y_2 - y_1|\right)^2
\end{equation*}

Aufgrund von $|a|^2 = a^2$ für jede Zahl $a\in\mathbb{R}$ wird die Ungleichung zu:
\begin{equation*}
(x_2 - x_1)^2 + (y_2 - y_1)^2 \leq (x_2 - x_1)^2 + 2|x_2 - x_1||y_2 - y_1| + (y_2 - y_1)^2
\end{equation*}
Vereinfacht ergibt sich:
\begin{equation*}
    0 \leq 2|x_2 - x_1||y_2 - y_1|
\end{equation*}
Der Ausdruck $2|x_2 - x_1||y_2 - y_1|$ ist immer positiv, da es sich um das Produkt von Beträgen handelt, die selbst positiv sind.

Daraus folgt, dass für alle möglichen Paare $(x_1, y_1)$ und $(x_2, y_2)$:
\begin{equation*}
    d_\text{eukl} \leq d_\text{okt}
\end{equation*}
Damit ist gezeigt, dass die Euklidische Distanz eine optimistische Heuristik darstellt.

% ============= Leerseite ============= %
\newpage
\pagestyle{empty}
\phantom{t}


%%%%%%%%%%%%%%%%%%%%%%%%%%%%%%%%%%%%%%%%%%%%%%%%%%%%%%%%%%%%%%
% ============= 	  Ende des Dokuments  	   ============= %
%%%%%%%%%%%%%%%%%%%%%%%%%%%%%%%%%%%%%%%%%%%%%%%%%%%%%%%%%%%%%%
\end{document}
