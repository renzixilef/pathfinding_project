%***********************************************************************************************%
%===============================================================================================%
% 																								%
%										Ausarbeitung zum Thema  								%
% 																								%
%               "Implementierung einer GUI-basierten Pathfinding-Algorithmen Benchmarking       %
%                            Software im Modul 'Effizient Programmieren'" 					    %
% 																								%
%									 von Felix Renzikowski, 2024							    %
% 																								%
%===============================================================================================%
%***********************************************************************************************%


% ============= Klassendefinition ============= %

\documentclass[
	%draft, 		% Entwurfsstadium
	final, 			% fertiges Dokument
	paper=a4,
	pagesize=auto,
	fontsize=12pt,
	ngerman,
	openright,
	twoside,		% Aktivieren, falls doppelseitiger Druck
	numbers=noendperiod,
]{scrreprt}

% ============= Verwendete Pakete ============= %

% Dokumentinformationen
\usepackage{hyperref}
\hypersetup{
	pdftitle={Implementierung einer GUI-basierten Pathfinding-Algorithmen Benchmarking Software im Modul 'Effizient Programmieren'},
	pdfsubject={Abschlussausarbeitung},
	pdfauthor={Felix Renzikowski},
	pdfkeywords={C++, CMake, Qt5, Effizient programmieren},
	pdfborder={0 0 0},
	colorlinks=true,
	breaklinks=true,
	linktocpage=true,
	citecolor=black,
	linkcolor=black,	
	menucolor=black,	
	urlcolor=black
}
\urlstyle{same}

% Allgemeine Pakete
\usepackage[T1]{fontenc}
\usepackage[utf8]{inputenc}	% Sonderzeichen
\usepackage{graphicx,color}
\usepackage[ngerman]{babel}
\usepackage{geometry}
\usepackage{amssymb}
\usepackage{amsmath}
\usepackage{acronym}
\usepackage[labelfont=bf]{caption}
\usepackage{tocloft}
\usepackage{enumitem}
\usepackage{float}
\usepackage[headsepline,automark,nouppercase]{scrlayer-scrpage}
\usepackage{pdfpages}
\usepackage{tikz, tikz-uml}
\usepackage{pgfplots}
\pgfkeys{/pgf/number format/.cd, use comma, 1000 sep={.}}
\usepgfplotslibrary{groupplots}
\usepackage{pgfgantt}
\usepackage{pdflscape}
\usepackage{afterpage}
\pgfplotsset{compat=newest}
\usepackage{grffile}
\usepackage{subcaption}
\usepackage[nottoc]{tocbibind}
\usepackage{lmodern}
\usepackage{siunitx}
\usepackage[export]{adjustbox}
\sisetup{output-decimal-marker = {,}}
\usepackage{mathtools}
\usepackage[mode=buildnew]{standalone}
\usepackage{wrapfig}
\usepackage[ngerman=ngerman-x-latest]{hyphsubst}
\raggedbottom 

\makeatletter
\renewcommand{\fnum@figure}{Abb. \thefigure}
\makeatother

% Einstellungen der Kopf- und Fußzeilen
\ohead{\headmark}									% Oben Rechts etwas angeben
\chead{}											% Keine Ausgabe oben in der Mitte
\cfoot[]{}											% Keine Seitenzahl in der Mitte
\ofoot[\pagemark]{\pagemark}						% Seitenzahlen rechts
%\automark{section}									% Sektionen in Kopfzeile (Einseitiger Modus)
\renewcommand*{\headfont}{\normalfont}				% Kein kursiver Text in Kopfzeile
\pagestyle{scrheadings}

% Absätze und Abstände
\renewcommand{\baselinestretch}{1}\normalsize		% Zeilenabstand
\setlength{\parindent}{0pt}							% Kein Einrücken nach Absatz
\geometry{inner=31.5mm,outer=31.5mm,bottom=3cm}		% Abstand zu unterem Seitenrand
\textwidth147mm

%%%%%%%%%%%%%%%%%%%%%%%%%%%%%%%%%%%%%%%%%%%%%%%%%%%%%%%%%%%%%%
% ============= 	 Beginn des Dokuments  	   ============= %
%%%%%%%%%%%%%%%%%%%%%%%%%%%%%%%%%%%%%%%%%%%%%%%%%%%%%%%%%%%%%%

\begin{document}
\pagenumbering{Roman}							% Römische Seitenzahlen bis zum Hauptteil
% ============= Titelseite =========== %
\newpage
\begin{center}
	\begin{Large}
	\textbf{Implementierung einer GUI-basierten Pathfinding-Algorithmen Benchmarking Software im Modul 'Effizient Programmieren'}\\
	\vspace{1cm}
	\textbf{Implementation of a GUI-based pathfinding algorithm benchmarking software in the module 'Effizient Programmieren'}
\end{Large}

\vspace{4cm}

Ausarbeitung\\
von\\
Felix Renzikowski

\vspace{4cm}

Durchgef\"uhrt am\\
\vspace{0.5cm}
\textbf{INSTITUT F\"UR AERODYNAMIK UND GASDYNAMIK}\\
\textbf{UNIVERSIT\"AT STUTTGART}\\

\vspace{2cm}

Dozent:\\
PD Dr. rer. nat. Manuel Keßler


\vspace{2cm}

Stuttgart, im September 2024
\end{center}
\addtocounter{page}{-1}

% ============= Vorspann ============= %
\pagestyle{scrheadings}
\renewcommand*{\chapterheadstartvskip}{\vspace*{2.3\baselineskip}}	% Abstand einstellen
\pdfbookmark[0]{Inhaltsverzeichnis}{toc}							% Inhaltsverzeichnis im pdf als Bookmark anzeigen
\renewcommand{\baselinestretch}{1.2}\normalsize						% Zeilenabstand
\tableofcontents													% Inhaltsverzeichnis
\renewcommand{\baselinestretch}{1.25}\normalsize					% Zeilenabstand
\renewcommand*{\chapterheadstartvskip}{\vspace*{2.3\baselineskip}}	% Abstand einstellen

% ============= Symbolverzeichnis ============= %
\chapter*{Nomenklatur}
\addcontentsline{toc}{chapter}{Nomenklatur}
\renewcommand{\baselinestretch}{1.1}\normalsize						% Zeilenabstand
%\begin{longtable}{p{0.2\textwidth} p{0.8\textwidth}}
%    \hline
%    Symbol& Bedeutung\\
%    \hline
%    \endhead
%    $A_{ES}$            & Eshelby-Tensor\\
%    $A_{MT}$            & Einflusstensor des Mori-Tanaka-Modells\\
%    $\boldsymbol{C}$	& Steifigkeitsmatrix eines Systems\\
%    $c$                 & größte Übertragungsgeschwindigkeit eines Systems\\
%    $CFL$               & Courant-Friedrichs-Lewy-Zahl\\
%    $\boldsymbol{D}$	& Dämpfungsmatrix eines Systems\\
%    $d$                 & charakteristische Länge des RVE\\
%    $E$                 & Elastizitätsmodul\\
%    $\boldsymbol{f}$	& Lastvektor eines Systems\\
%    $f$	                & Fließfläche\\
%    $G$                 & Schubmodul\\
%    $\boldsymbol{I}$    & Einheitsmatrix\\
%    $K$                 & Kompressionsmodul\\
%    $k_i$               & Faktoren in \textit{MAT\_187}\\
%    $l$                 & charakteristische Länge der Inhomogenität\\
%    $L_{f}$         & Länge einer Faser\\
%    $L_{makro}$         & Ausdehnung der Makroebene\\
%    $\boldsymbol{M}$	& Massematrix eines Systems\\
%    $p$                 & mittlere Normalspannung\\
%    $R$                 & Festigkeit\\
%    $r_{f}$              & Faserradius\\
%    $\delta_{RVE}$      & Faktor auf die Größe des RVE\\
%    $\boldsymbol{S}$	& Nachgiebigkeitsmatrix eines Systems\\
%    $\Delta t$	        & diskreter Zeitschritt\\
%    $\boldsymbol{u}$	& Verschiebungsvektor eines Systems\\
%    $\boldsymbol{x}^{makro}$         & Ortsvektor eines RVE\\
%    $\Delta x_{min}$    & kleinster Abstand zwischen diskreten Knoten\\
%    $\gamma$            & Verzerrung\\
%    $\varepsilon$          & Dehnung\\
%    $\zeta$             & Faktor im Halpin-Tsai-Modell\\
%    $\eta$              & Faktor im Halpin-Tsai-Modell\\
%    $\nu$               & Poisson-Zahl/Querkontraktionszahl\\
%    $\theta$            & Faserwinkel zur Rotation um die y-Achse\\
%    $\nu_{\parallel\perp}$ & kleine Poisson-Zahl\\
%    $\nu_{\perp\parallel}$ & große Poisson-Zahl\\
%    $\sigma$            & Normalspannung\\
%    $\sigma_{VM}$       & von-Mises-Vergleichsspannung\\
%    $\tau$              & Schubspannung\\
%    $\phi$              & Faserwinkel zur Rotation um die z-Achse\\
%    $\varphi$           & Faservolumengehalt\\
%    \caption*{}
%\end{longtable}

\begin{table}[H]
    \begin{tabular}{ll}
        \hline
        Abkürzung\phantom{123456}& Bedeutung \phantom{1234567891234567891234567891234567899}\\
        \hline
    \end{tabular}
\end{table}
\begin{addmargin}[0.2cm]{0cm}
    \vspace{-0.6cm}
    \begin{acronym}[MMAE~~~~~~~~~~~~]
        \setlength{\itemsep}{-\parsep}
        \acro{JPS}{Jump-Point-Search}
        \acro{IDE}{Integrated Development Environment}
        \acro{CI}{Continuous Integration}
        \acro{GUI}{Graphical User Interface}
    \end{acronym}
\end{addmargin}
\renewcommand{\baselinestretch}{1.25}\normalsize					% Zeilenabstand
\markleft{Nomenklatur}

%\renewcommand*{\listoffigures}{
%	\begingroup
%	%\tocchapter
%	\tocfile{\listfigurename}{lof}
%	\endgroup
%}

%\cleardoublepage
%\setlength{\cftfigindent}{0em}
%\renewcommand{\baselinestretch}{1.1}\normalsize						% Zeilenabstand
%\pdfbookmark{Abbildungsverzeichnis}{lof}
%\listoffigures 														% Abbildungsverzeichnis
%\markleft{Abbildungsverzeichnis}

%\renewcommand*{\listoftables}{
%	\begingroup
	%\tocchapter
%	\tocfile{\listtablename}{lot}
%	\endgroup
%}

\pdfbookmark[0]{Tabellenverzeichnis}{tbv}							% Tabellenverzeichnis im pdf als Bookmark anzeigen
\setlength{\cfttabindent}{0em}
\listoftables														% Tabellenverzeichnis
\renewcommand{\baselinestretch}{1}\normalsize						% Zeilenabstand

% ============= Hauptteil ============= %
\cleardoubleoddpage

\pagenumbering{arabic}  											% Arabische Zahlen im Hauptteil bis zum Anhang

%********************************
\chapter{Einleitung}
\label{ch:einleitung}
%********************************

%********************************
\section{Zielsetzung}
\label{sec:motivation}
%********************************
% TODO: Inhalt...

%********************************
\section{Gliederung der Arbeit}
\label{sec:gliederung}
%********************************


%\include{02_grundlagen}

%%********************************
\chapter{Programmaufbau und -struktur}
\label{ch:aufbau}
%********************************

%********************************
\section{Aufbau des Backends}
\label{sec:aufbau_backend}
%********************************
In diesem Abschnitt soll ein Überblick über den Aufbau und die Strukturierung des Backends gegeben werden.
`Backend' bezeichnet hierbei den ausführenden Teil des Programms, also Objekte und Strukturen die zur Berechnung/Lösung des Grids notwendig sind und keinen direkten User-Kontakt haben.

%********************************
\subsection{Überblick}
\label{subsec:aufbau_backend_ueberblick}
%********************************
Das Backend ist in zwei Namespaces unterteilt: `GridGenerator' und `Pathfinder'.
Ersterer beinhaltet jegliche Funktionalität die zur Erstellung der Grid-Objekte, zum Platzieren der Hindernisse und zur Positionierung des Start- und Endpunkts notwendig sind.
Der Namespace `Pathfinder' gruppiert Komponenten die zur Berechnung und Lösung ebendieser Objekte notwendig sind, sowie Funktionen zur Überwachung dieses Prozesses.
Im Folgenden soll zunächst auf die Erstellung der Grid-Objekte und sodann auf den Namespace `Pathfinder' eingegangen werden.
%********************************
\subsubsection{Namespace: GridGenerator}
\label{subsubsec:aufbau_backend_ueberblick_gridgen}
%********************************
Dieser Namespace setzt sich zusammen aus den Klassen \texttt{Cell}, \texttt{Grid} und \texttt{Obstacle\-Generator}, sowie einigen Subklassen, unterstützenden Datenstrukturen und Enumerationen.
\begin{itemize}
    \item \textttt{Cell} \\
    \begin{figure}[H]
        \vspace{-0.5cm}
        \centering
        \includestandalone[width=0.7\textwidth]{assets/uml/cell_class_uml}
        \caption{UML-Klassendiagramm der Klasse \texttt{Cell}}
        \label{fig:uml_cell}
    \end{figure}
    Die \texttt{Cell}-Klasse hält Informationen und Funktionalität einer spezifischen Gridzelle.
    Jede Zelle hält ihre Kosten (\texttt{CellCost}) und einen Zustand (\texttt{Cell\-State}: Hindernis, begehbar, etc.).
    Außerdem enthält die \texttt{Cell}-Klasse mehrere Helferfunktionen für die Interaktion mit den Gitterzellen.

    \item \textttt{Grid} \\
    \begin{figure}[H]
        \vspace{-0.5cm}
        \centering
        \includestandalone[width=0.7\textwidth]{assets/uml/grid_class_uml}
        \caption{UML-Klassendiagramm der Klasse \texttt{Grid}}
        \label{fig:uml_grid}
    \end{figure}
    Die \texttt{Grid}-Klasse hält Informationen und Funktionalität die Interaktionen mit dem Grid ermöglichen.
    Das tatsächliche Grid ist als \textttt{vector<vector<Cell> >} aufgebaut, der Zugriff findet über den \texttt{()}
    Operator und die Struktur \texttt{GridCoordinate} statt.
    Bei der Instantiierung eines \texttt{Grid}-Objekts wird eine Referenz zu einem \texttt{ObstacleGenerator}-Objekt (ähnlich eines Decorator-Pattern \cite{nesteruk2021})
    welchem die Verantwortung für die Hindernisgenerierung und das Festsetzen des Start- und Endpunkts übergeben wird.
    \newpage
    \item \textttt{ObstacleGenerator} \\
    \begin{figure}[H]
        \vspace{-0.5cm}
        \centering
        \includestandalone[width=0.7\textwidth]{assets/uml/obstacleGen_class_uml}
        \caption{UML-Klassendiagramm der Klasse \texttt{ObstacleGenerator}}
        \label{fig:uml_obstacle_gen}
    \end{figure}
    Die \texttt{ObstacleGenerator}-Klasse und ihre Subklassen sind nach einem Strategy-Pattern aufgebaut \cite{nesteruk2021}.
    Übergreifende Funktionalität der verschiedenen Algorithmen zur Erstellung der Hindernisse wird in der Elternklasse \texttt{Obstacle\-Generator} gehalten.
    Zudem wird hier eine virtuelle Methode \texttt{generate\-Obstacles()} definiert, die von den Subklassen mit dem jeweiligen Algorithmus überschrieben wird.
    Für den Kontext ist der tatsächliche Algorithmus dadurch irrelevant, es kann mit einem generischen Objekt vom Typ \texttt{ObstacleGenerator} gearbeitet werden.
    Zusätzlich wird in der gleichen Header-Datei ein \texttt{enum} und ein dazugehöriger Parser-\texttt{struct} für die verschiedenen Algorithmen definiert.
\end{itemize}
%********************************
\subsubsection{Namespace: Pathfinder}
\label{subsubsec:aufbau_backend_ueberblick_pathfinder}
%********************************
Der Namespace `Pathfinder' beeinhaltet die Klasse \texttt{PathfindingParent} und ihre Subklassen, sowie die Klasse \texttt{PathfinderTimer}.
Zusätzlich werden über diverse \texttt{structs} und \texttt{enums} Zustände und Interaktionen mit dem Frontend definiert.
\begin{itemize}
    \item \texttt{PathfindingParent}\\
    Die \texttt{PathfindingParent}-Klasse und ihre Subklassen sind ähnlich wie die \textttt{ObstacleGen}-Klasse nach einem Strategy-Pattern aufgebaut \cite{nesteruk2021}.
    In der Elternklasse wird ein generischer Pathfinder initialisiert, der konkrete Algorithmus ist auch hier vor dem Kontext verborgen.
    Der Instanz einer \texttt{Pathinding\-Parent}-Subklasse wird bei Konstruktion eine Referenz zu einem \texttt{Grid}-Objekt übergeben.
    Dieses Objekt repräsentiert den Handlungsrahmen des jeweiligen Pathfinding-Algorithmus.
    Über die virtuelle Methode \texttt{nextStep()} kann nach Initialisierung der nächste Schritt des jeweiligen Algorithmus durchgeführt werden.
    Zu Debugging-Zwecken wurde außerdem eine Methode \\ \texttt{solveNoWait()} implementiert, die bis zum Ende des Lösungsprozesses
    iterativ die \texttt{nextStep()}-Methode aufruft.\\
    Für die diversen Subklassen wird auch hier noch ein zusätzlicher Parsing-\texttt{struct} und -\texttt{enum} für Interaktionen des Frontends definiert.
    \begin{figure}[H]
              \vspace{-0.5cm}
              \centering
              \includestandalone[width=0.7\textwidth]{assets/uml/pathfindingParent_class_uml}
              \caption{UML-Klassendiagramm der Klasse \texttt{PathfindingParent}}
              \label{fig:uml_pathfinder}
    \end{figure}
    \item \texttt{PathfinderTimer}\\
    \begin{figure}[H]
        \vspace{-0.5cm}
        \centering
        \includestandalone[width=0.7\textwidth]{assets/uml/pathfindingTimer_class_uml}
        \caption{UML-Klassendiagramm der Klasse \texttt{PathfinderTimer}}
        \label{fig:uml_pathfinder_timer}
    \end{figure}
    Die Klasse \texttt{PathfinderTimer} implementiert Metriken und Funktionalität zum rechenzeitbasierten Benchmarking der \texttt{PathfinderParent}-Subklassen\-instanzen.
\end{itemize}
%********************************
\subsection{Algorithmen und Konzepte}
\label{subsec:aufbau_backend_konzepte}
%********************************
\subsubsection{Hindernisgenerierung}
In den \texttt{ObstacleGenerator}-Subklassen werden verschiedene Algorithmen definiert, die zur prozeduralen Generierung von
Hindernissen auf den 2D-\texttt{Grid}-Objekten eingesetzt werden.
So definiert die Subklasse \texttt{RandomObstacleGenerator} einen Algorithmus der bis zur gewünschten \texttt{obstacleDensity}
pseudozufällig Hindernisse auf dem Gitter verteilt.
In diesem Abschnitt soll allerdings kurz auf die in der Literatur bekannten Algorithmen `Drunken Walk' und `Perlin Noise' eingegangen werden.
Diese Algorithmen werden auch im Design von Computerspielen eingesetzt um prozedural Level zu generieren \cite{koesnaedi2022, andrian2023},
was einen sehr ähnlichen Use-Case zu diesem Projekt darstellt.\\\\

\textbf{Drunken Walk}\\
Der `Drunken Walk'-Algorithmus (auch `Random Walk' oder `Drunkard's Walk'), stellt einen simplen aber effektiven Algorithmus
zur Erstellung von zufälligen Pfaden auf einem Gitter beliebiger Dimension dar.
Mathematisch betrachtet bildet dieser Algorithmus einen stochastischen Prozess, welcher eine Spur bestehend aus
einer Folge von Zufallsschritten beschreibt ab. \cite{pearson1905}
In der vorliegenden Implementation wird dafür zunächst das komplette Gitter mit Hindernissen gefüllt, anschließend wird
eine zufällige Zelle für den Startpunkt des Pfades ausgewählt.
Daraufhin wird mit zufälligen Schritten (basierend auf einem \textit{Mersenne Twister}-Generator) so lange ein begehbares Gebiet aus den Hindernissen heraus``gemeißelt'', bis der
spezifizierte Hindernisanteil erreicht ist.
\lstinputlisting[language=C++, firstline=1, caption={Drunken Walk-Algorithmus},label={lst:obstacle_drunken_walk}]
    {assets/listings/obstacle_drunken_walk.txt}
Der `Drunken Walk'-Algorithmus findet außerdem zahlreiche Anwendungen in Gebieten wie der Physik, Psychologie oder auch Wirtschaftswissenschaft. \cite{weiss1982, nosofsky1997, kodde1984}\\\\

\textbf{Perlin Noise}\\
`Perlin Noise' ist eine pseudozufällige Rauschfunktion basierend auf Gradientenwerten in einem äquidistanten Gitter.
Diese Funktion wird häufig zur Bildsynthese zur Generierung natürlicher Texturen für z.B. Wolken oder Gewässer eingesetzt.
Zudem findet sie auch in vielen modernen Computerspielen (beispielsweise \textit{Minecraft} (Microsoft)) Anwendung zur
prozeduralen Generierung von Landschaftstopologien.

\subsubsection{Finden des kürzesten Pfades}
In den Subklassen der \texttt{PathfinderParent}-Klasse werden verschiedenen Algorithmen implementiert um für gegebene
\texttt{Grid}-Objekte einen möglichen Pfad von Startpunkt zu Endpunkt zu identifizieren.
Die eingesetzten Algorithmen unterscheiden sich hauptsächlich im Hinblick auf die Laufzeiteffizienz (Dauer eines Berechnungsschritts)
und die Ressourceneffizienz (Anzahl der untersuchten \texttt{Cell}-Objekte während der kompletten Berechnung).
Im Folgenden soll zunächst auf allgemeine Überlegungen für die Implementierung dieser Algorithmen und danach kurz auf
jeden eingesetzten Algorithmus im Detail eingegangen werden.\\\\

\textbf{Allgemeine Überlegungen}\\
Die in diesem Projekt eingesetzten Algorithmen benötigen alle eine über die Rechenschritte persistente Datenstruktur,
welche die \texttt{Cell}-Objekte hält, die als nächstes untersucht werden müssen.
Da in jedem der Algorithmen eine konstante Entwicklung des Pfades vom Startpunkt aus durchgeführt wird, wächst diese
Datenstruktur mit jedem Berechnungsschritt und muss demnach dynamisch angelegt sein.
Die Performanz der Algorithmen hängt stark von der Effizienz des Zugriffs auf und der Manipulation dieser Datenstruktur an bestimmten Indices ab.
Ein gutes Datenmanagement und optimierte Datenstrukturen können somit die Laufzeiteffizienz erheblich steigern.
Ein sorgfältiges Ausbalancieren zwischen Speichernutzung und Rechenleistung ist elementar, da eine zu hohe Speicherkomplexität
das System vor allem für sehr große Gitter stark belasten könnte.
Bei der Implementierung von Wegfindungsalgorithmen in C++ stehen uns eine Vielzahl von Datenstrukturen zur Verfügung,
die jeweils ihre eigenen Vor- und Nachteile mitbringen. \cite{iso2020}
\begin{itemize}
    \item \texttt{std::vector}\\
    Eine gängige Wahl ist \texttt{std::vector}.
    Diese Struktur bietet schnellen Zugriff ($\mathcal{O}(1)$) auf Elemente durch direkte Indizierung, allerdings ist
    das Einfügen und Entfernen von Elementen in der Mitte des Vektors eine kostspieligere Operation ($\mathcal{O}(n)$).
    \item \texttt{std::list}\\
    \texttt{std::list}, eine doppelt verkettete Liste, erlaubt schnelles Einfügen und Entfernen von Elementen ($\mathcal{O}(1)$),
    da lediglich Zeiger umgestellt werden müssen.
    Der Zugriff auf ein bestimmtes Element ist allerdings langsamer und benötigt im Allgemeinen eine Durchquerung der Liste (($\mathcal{O}(n)$)).
    \item \texttt{std::dequeue}\\
    Diese Datenstruktur bietet ähnliche Eigenschaften wie \texttt{std::vector}, erlaubt aber zusätzlich das Einfügen
    und Entfernen von Elementen in konstanter Zeit ($\mathcal{O}(1)$) am Anfang, wie auch am Ende des Containers.
    \item \texttt{std::priority\_queue}\\
    \texttt{std::priority\_queue} (Implementierung eines \textit{Binary Heap}) ermöglicht das Einfügen von Elementen und das Extrahieren des größten Elements jeweils
    mit Komplexität $\mathcal{O}(n\cdot\log n)$.
    Die zugrundeliegende Datenstruktur ist auch hier normalerweise ein \texttt{std::vector}, kann aber bei Bedarf angepasst werden.
    Für die Objekte, die von der \texttt{std::priority\_queue} gehalten werden, muss eine Methode zur Bestimmung des
    größeren zweier Elemente definiert sein.
    \item \texttt{std::set} und \texttt{std::map}\\
    \texttt{std::set} und \texttt{std::map} erlauben das Einfügen, Löschen und Suchen von Elementen in $\mathcal{O}(\log n)$ Zeit
    dank ihrer Implementierung als balancierte binäre Suchbäume.
    In der Praxis kann jedoch die konstante Faktorzeit erheblich sein, aufgrund der Rot-Schwarz-Baum-Balancierungsschritte.
    \item \texttt{std::unordered\_set} und \texttt{std::unordered\_map}\\
    \texttt{std::unordered\_set} und \texttt{std::unordered\_map}, die hashbasierte Datenstrukturen sind, bieten im
    Idealfall konstante Zeit ($\mathcal{O}(1)$) für Einfügen, Löschen und Suchen, jedoch kann dies im Worst-Case auf $\mathcal{O}(n)$
    ansteigen, z.B. bei ungünstigen Hashfunktionen.
\end{itemize}
Für die Anforderungen von Pfadfindealgorithmen, insbesondere von Algorithmen wie A* und Dijkstra's Algorithmus,
stellt die Datenstruktur \texttt{std::priority\_queue} in diesem Projekt eine optimale Wahl dar.
Diese Algorithmen arbeiten nach einem `Greedy'-Ansatz \cite{korte2006}, indem sie immer den Pfad mit den geringsten angesammelten Kosten wählen.
Daher ist eine Datenstruktur, die einen effizienten Zugriff auf das Element mit der höchsten Priorität (in diesem Fall
das Element mit den geringsten Kosten) ermöglicht, von großem Nutzen.\\
Bei der Verwendung einer \texttt{std::priority\_queue} ist es, wie bereits erwähnt, entscheidend, eine geeignete Vergleichsfunktion
zu implementieren, um die Priorität der Elemente zu bestimmen.
Im Kontext von Pfadfindealgorithmen wäre diese Funktion typischerweise so gestaltet, dass sie die geringsten angesammelten
Kosten als höchste Priorität behandelt.
In diesem Projekt werden Objekte vom primitiven Typ \texttt{GridCoordinate} gehalten (siehe \ref{subsubsec:aufbau_backend_ueberblick_gridgen}),
daher wurde die Vergleichsmethode wie folgt implementiert:
\lstinputlisting[language=C++, firstline=1, caption={Vergleichsmethode für die \texttt{std::priority\_queue}},
label={lst:queue_compare_method}]{assets/listings/priority_queue_compare_method.txt}
Diese Implementation ermöglicht den Erhalt des Kontexts durch Rückgabe einer Lambda-Funktion von \texttt{compareCells()}.
Der Erhalt des Kontextes spielt hier eine wichtige Rolle, da \texttt{compareCells()} eine Instanzfunktionalität der
\texttt{Grid}-Klasse ist und über den \texttt{(*this)(GridCoordinate\&)}-Operator auf die jeweiligen \texttt{Cell}-Objekte
zugegriffen wird.\\
Zusätzlich zur Optimierung der Datenstrukturen kann die Laufzeiteffizienz auch durch eine gute Vorverarbeitung der Daten,
zum Beispiel durch eine geeignete kartografische Abstraktion oder das Einbringen von Heuristiken, optimiert werden.
Diese Techniken können jedoch unter Umständen die Qualität der Lösung beeinträchtigen, sodass immer ein Kompromiss zwischen
der Qualität der Lösung und der erforderlichen Berechnungszeit gefunden werden muss.
Auf die verwendeten Heuristiken soll im Folgenden noch genauer eingegangen werden.\\
Mathematisch betrachtet basieren alle hier implementierten Algorithmen \cite{morin1982} auf dem Optimalitätsprinzip von Bellman, nach
dem sich eine optimale Lösung aus optimalen Teillösungen zusammensetzt:
\begin{quotation}
An optimal policy has the property that whatever the initial state and initial decision are, the remaining decisions
must constitute an optimal policy with regard to the state resulting from the first decision.\\
-- BELLMANN, 1957 \cite{bellman2010}
\end{quotation}
Im Kontext der implementierten Algorithmen lässt sich das folgendermaßen ausdrücken:
Angenommen, zwischen Knoten $v$ und Startknoten $s$, existiert der Pfad $P$ von $s$ zu $v$ mit $P = [s, \dots, v]$.
Wenn nun ein Knoten $u$ und eine Kante $(v, u)$ eingefügt werden, um einen neuen Pfad $P' = [s, \dots, v, u]$ zu
bilden, dann ist $P'$ der kürzeste Pfad von $s$ zu $u$, wenn $P$ der kürzeste Pfad von $s$ zu $v$ war. \\\\

\textbf{Distanzberechnung und Heuristiken}\\
Bei Pathfinding-Algorithmen spielen Distanzberechnung und Heuristiken eine entscheidende Rolle, um effiziente und optimale Pfade zu finden.
Die Distanzberechnung wird verwendet, um die tatsächliche Entfernung zwischen zwei Punkten auf einem Gitter zu bestimmen.
Eine gängige Methode hierfür ist die Oktile-Distanz, die nicht nur horizontale und vertikale, sondern auch diagonale Schritte berücksichtigt.
Für die Oktile-Distanz zwischen zwei Punkten $(x_1, y_1)$ und $(x_2, y_2)$ gilt:
\begin{equation}
    \label{eq:distance_octile}
    d_\text{okt}(P_1, P_2) = \max(|x_2 - x_1|, |y_2 - y_1|) + ( \sqrt{2} - 1) \min(|x_2 - x_1|, |y_2 - y_1|)
\end{equation}
Für die Heuristiken, die eine Schätzung der Entfernung von einem Punkt zum Ziel darstellen und somit die Suchrichtung
des Algorithmus lenken, wird oft die Euklidische Distanz verwendet.
Die Euklidische Distanz misst die direkte Luftlinie zwischen zwei Punkten und bietet eine präzise, jedoch rechenintensive Heuristik.
Für die Euklidische Distanz zwischen zwei Punkten $(x_1, y_1)$ und $(x_2, y_2)$ gilt:
\begin{equation}
    \label{eq:distance_euklid}
    d_\text{eukl}((x_1, y_1), (x_2, y_2)) = \sqrt{(x_2 - x_1)^2 + (y_2 - y_1)^2}
\end{equation}

\textbf{Dijkstra's Algorithmus}\\
Dijkstra's Algorithmus, benannt nach seinem Entwickler Edsger W. Dijkstra, ist ein prominenter Vertreter der Graphenalgorithmen,
speziell der kürzesten-Pfad-Algorithmen. \cite{dijkstra1959}\\
Der Algorithmus funktioniert folgendermaßen: Zunächst wird dem Startknoten der Abstandswert $0$ zugewiesen.
Für den aktuellen Knoten betrachtet man nun dessen ungeprüfte Nachbarknoten und summiert den Abstandswert des aktuellen
Knotens und das Gewicht der Kante, die zum Nachbarknoten führt (Distanz zum Nachbarknoten).
Diese Summe wird auch als $G(n)$ (G-Kosten) bezeichnet.
Wenn die eben berechnete Summe kleiner ist als der aktuelle Wert für $G(n)$ des Nachbarknotens, aktualisiert man diesen Wert.
Wenn alle Nachbarn des aktuellen Knotens geprüft wurden, kennzeichnet man ihn als geprüft.
Anschließend wird der Knoten mit dem geringsten vorläufigen Wert, der noch nicht geprüft wurde, als neuer aktueller Knoten gewählt.
Dieser Vorgang wird wiederholt, bis der zu prüfende Knoten der Endknoten ist.
Am Ende des Algorithmus ist/sind der kürzeste Pfad/die kürzesten Pfade von Startknoten zu Endknoten bekannt.
Die Laufzeit von Dijkstra's Algorithmus ist im Allgemeinen abhängig von der Implementierung der verwendeten Datenstrukturen
und des Abstandes von Start- und Endknoten. \cite{dijkstra1959}
Im schlechtesten Fall ist die Zeitkomplexität $\mathcal{O}((V+E)\cdot\log V)$, wobei $V$ die Menge der Knoten und $E$
die Menge der Kanten des Graphen ist. \cite{cormen2022}\\\\

\textbf{A* Algorithmus}\\
Der A*-Algorithmus ist, ähnlich wie Dijkstra's Algorithmus, ein prominenter Vertreter der Graphenalgorithmen und speziell
der kürzesten-Pfad-Algorithmen.
Entwickelt von Peter Hart, Nils Nilsson und Bertram Raphael, löst er das Problem des kürzesten Pfads von einem Startknoten
zu einem Zielknoten in einem gewichteten Graphen mit nichtnegativen Kantengewichten. \cite{hart1968}\\
Der A*-Algorithmus ist eine Erweiterung des Dijkstra-Algorithmus, der zusätzlich eine sogenannte Heuristik verwendet,
um die Suche zum Zielknoten zu leiten.
Bei der A*-Methode werden zu jedem Knoten zunächst, wie bei Dijkstra, Abstandswerte oder $G(n)$ berechnet.
Die Besonderheit ist jedoch, dass zusätzlich die erwartete Reststrecke zum Zielknoten, auch $H(n)$ (H-Kosten) genannt, berücksichtigt wird.
Die Gesamtstrecke, auch $F(n)$ (F-Kosten), setzt sich also aus dem aktuellen Abstandswert und der heuristisch geschätzten
Distanz vom betrachteten Knoten zum Ziel zusammen.
Anstatt den Knoten mit den geringsten G-Kosten (wie bei Dijkstra) als nächsten Knoten auszuwählen, wird der
Knoten mit den geringsten F-Kosten als nächster zu überprüfender Knoten gewählt.
Dies ermöglicht A*, effizienter als Dijkstra den kürzesten Weg zu einem gegebenen Zielknoten zu finden, ohne jeden
möglichen Knoten betrachten zu müssen.\\
Damit ist A* besonders geeignet für Probleme, bei denen der Zielzustand bekannt ist, sodass eine sinnvolle Heuristik angewendet werden kann. \cite{hart1968}
Eine Heuristik für den A*-Algorithmus sollte die tatsächlichen Kosten zum Ziel nie überschätzen, eine solche Heuristik
wird als optimistisch bezeichnet (ein Beweis, dass dies für die euklidische Distanz auf den vorliegenden Gittern gilt, ist in Abschnitt \ref{sec:beweis_heuristik} zu finden).
Ist die Heuristik optimistisch, kann garantiert werden, dass der A*-Algorithmus immer die optimale Lösung findet.
Im Gegensatz dazu, wenn eine Heuristik die tatsächlichen Kosten überschätzt, kann der A*-Algorithmus eine suboptimale
Lösung finden, obwohl er möglicherweise schneller eine Lösung findet als Dijkstra's Algorithmus.\\
Im schlechtesten Fall, wenn die Heuristik keine nützliche Information liefert (zum Beispiel immer null zurückgibt, was den
A*-Algorithmus in den Dijkstra-Algorithmus umwandelt), ist die Laufzeit auch hier $\mathcal{O}((V+E)\cdot\log V)$.\\\\

\textbf{Jump Point Search-Algorithmus}\\
Der  \ac{JPS}-Algorithmus, entwickelt von Daniel Harabor und Alban Grastien im Jahr 2011, ist eine Optimierung des
A*-Algorithmus speziell für uniforme Raster mit orthogonalen Bewegungen, wie sie häufig in Videospielen und ähnlichen
Simulationen zu finden sind. \cite{harabor2011}\\
\ac{JPS} beschleunigt A*, indem unnötige Knoten übersprungen werden und nur an den Stellen gesucht wird, die als ``Sprungpunkte'' bezeichnet werden.
Sprungpunkte sind Punkte, von denen aus der Pfad zu einem benachbarten Punkt gebogen werden könnte, um einen kürzeren Pfad zu erzielen.
Bei der Begutachtung der Nachbarn eines Knotens nimmt JPS nicht alle in die offene Liste auf (wie es A* tun würde),
sondern nur diejenigen, die als Sprungpunkte angesehen werden.
Dadurch kann JPS die Anzahl der zu berücksichtigenden Knoten erheblich reduzieren, was zu einer erheblichen Beschleunigung
gegenüber dem Standard-A*-Algorithmus führen kann.
Es ist zu beachten, dass JPS am effizientesten ist, wenn die Bewegungskosten im Raster einheitlich sind und nur
orthogonale Bewegungen erlaubt sind.
Bei nicht einheitlichen Kosten oder diagonalen Bewegungen ist A* in der Regel effizienter.
In Bezug auf die Zeitkomplexität kann JPS merklich effizienter als der A*-Algorithmus sein, die genaue Big-O-Notation
kann jedoch je nach Art des verwendeten Rasters und anderen Faktoren variieren.
Unter idealen Bedingungen, Suchen über große Raster mit vielen offenen Bereichen, kann JPS die Suchzeit im Vergleich
zu A* um einen Faktor von bis zu 10 reduzieren. \cite{harabor2014}

%********************************
\section{Aufbau des Frontends}
\label{sec:aufbau_frontend}
%********************************
Im Folgenden soll ein kurzer Überblick über den Aufbau des Frontends und des Interfaces zwischen Front- und Backend gegeben werden.
Das Frontend beinhaltet hierbei vor allem Funktionalität der \ac{GUI} (Widgets, Fenster, Layouts, etc.) und des \ac{CLI}.
Zunächst wird allerdings eine Übersicht über die Interfaceklassen gegeben.
%********************************
\subsection{Überblick - Interfaces}
\label{subsec:aufbau_frontend_ueberblick_interfaces}
%********************************
Die Funktionalität des Interfaces für die Benutzung des Programms mit \ac{GUI} beschränkt sich auf die Klasse \texttt{RunnerParent}
und ihre Subklassen \texttt{SingleRun} und \texttt{MultiRun}.
Direkt nach Initialisierung dieser Klassen werden die resultierenden Objekte auf einen neuen Thread verschoben, um den Event-Loop der \ac{GUI} nicht zu blockieren.
Zur Kommunikation mit der \ac{GUI} wird das threadsichere \texttt{SIGNAL-SLOT}-Prinzip \cite{Qt52024} des \textit{Qt5}-Frameworks eingesetzt.
Außerdem wird für die Benutzung als \ac{CLI} die Schnittstelle \texttt{HeadlessRunner} definiert.
\begin{itemize}
    \item \texttt{RunnerParent}\\
    \begin{figure}[H]
        \vspace{-0.5cm}
        \centering
        \includestandalone[width=0.7\textwidth]{assets/uml/runnerParent_class_uml}
        \caption{UML-Klassendiagramm der Klasse \texttt{RunnerParent}}
        \label{fig:uml_runner_parent}
    \end{figure}
    Die Subklassen der \texttt{RunnerParent}-Klasse bieten Funktionalität für einzelne Durchläufe über Instanzen der \texttt{SingleRun}-Klasse
    oder Benchmarking-Durchläufe mit mehreren Gridkonfigurationen und Lösealgorithmen über die \texttt{MultiRun}-Klasse.

    \item \texttt{HeadlessRunner}\\
    \begin{figure}[H]
        \vspace{-0.5cm}
        \centering
        \includestandalone[width=0.7\textwidth]{assets/uml/headlessRunner_class_uml}
        \caption{UML-Klassendiagramm der Klasse \texttt{HeadlessRunner}}
        \label{fig:uml_headlessRunner}
    \end{figure}
    Die Klasse \texttt{HeadlessRunner} enthält neben der allgemeinen Schnittstellenfunktionalität von \ac{GUI}-Seite aus,
    einige statische Helfermethoden für Formatierung und Darstellung von \texttt{QString}-Objekten.
    Die Klassen \texttt{Single\-RunDialog} und \texttt{MultiRunDialog} erben von \texttt{HeadlessRunner}, die allgemeine Funktionalität,
    welche auch für die Darstellung mit \ac{GUI} relevant ist.
    Auf diese beiden Klassen soll allerdings später noch genauer eingegangen werden.

    \item \texttt{PathfindingCommandParser}
    \begin{figure}[H]
        \vspace{-0.5cm}
        \centering
        \includestandalone[width=0.7\textwidth]{assets/uml/pathfindingCommandParser_class_uml}
        \caption{UML-Klassendiagramm der Klasse \texttt{PathfindingCommandParser}}
        \label{fig:uml_commandlineparser}
    \end{figure}
    Im Namespace \texttt{Application} befindet sich neben der eben erläuterten \\ \texttt{HeadlessRunner}-Superklasse auch die
    Funktionalität des Command-Line-Parsers.
    Diese Klasse verarbeitet die etwaigen Command-Line-Argumente bei Ausführen der Applikation über das Terminal.
    Die verschiedenen möglichen Flaggen die gesetzt werden können, haben jeweils eine eigene Beschreibung, die über
    das \ac{CLI} übliche \texttt{-h} (\texttt{--help}) Argument abgerufen werden können.
\end{itemize}
%********************************
\subsection{Überblick - GUI}
\label{subsec:aufbau_frontend_ueberblick_gui}
%********************************
Die Funktionalitäten des \ac{GUI} sind aufgeteilt auf die Namespaces \texttt{GUI} und den Subnamespace \texttt{GUI::Widgets}.
Aus Gründen der Übersichtlichkeit wird für die hier implementierten GUI-Klassen kein UML-Diagramm bereitgestellt.
Allerdings soll hier kurz auf das \textit{Qt5}-Framework, dessen Mechanismen und deren vorliegende Implementierung eingegangen werden.\\
Das Qt5 Framework ist eine leistungsfähige Bibliothek zur Entwicklung plattformübergreifender Anwendungen und GUIs.
Es wird in C++ entwickelt und bietet Bindings für verschiedene andere Programmiersprachen wie Python, Ruby und Java.
Eine der herausragenden Eigenschaften von Qt5 ist seine Fähigkeit, aus einer einzigen Codebasis für mehrere Plattformen zu kompilieren, einschließlich Windows, macOS, Linux, iOS und Android.
Dies ermöglicht Entwicklern, ohne tiefgreifende Änderungen am Code Anwendungen für verschiedene Betriebssysteme bereitzustellen.
Unter der Haube bietet Qt5 Module für alles Mögliche: von GUI-Entwicklung über Netzwerkkommunikation bis hin zu Multimedia und Datenbanken. \cite{Qt52024}\\
Ein zentrales Konzept von Qt ist der \texttt{SIGNAL}-\texttt{SLOT}-Mechanismus, der eine elegante Lösung für die Implementierung von ereignisgesteuerten Architekturen bietet.
Signale und Slots ermöglichen die Kommunikation zwischen Objekten: Ein Signal wird von einem Objekt ausgesendet und kann mit einem oder mehreren Slots (Methoden) verbunden werden, die dann auf das Signal reagieren.
Dieser Mechanismus ist dabei von C++'s nativer Pointer-Mechanik inspiriert, jedoch weitaus sicherer und flexibler.
Ein großer Vorteil des \texttt{SIGNAL}-\texttt{SLOT}-Mechanismus ist seine Thread-Sicherheit.
\textit{Qt5} sorgt dafür, dass die Signal-Emission und Slot-Ausführung, selbst wenn sie von verschiedenen Threads ausgeführt werden, ohne race conditions oder undefiniertes Verhalten stattfinden.
Dies geschieht durch die Verwendung von Queued Connections, welche die Emitter-Thread-Aufrufe in der Event-Queue des Empfänger-Threads platzieren, ohne dass manuelle Sperrmechanismen wie \texttt{mutex} erforderlich sind.
Dadurch wird eine klare Trennung und Synchronisation zwischen unterschiedlichen Ausführungskontexten ermöglicht und die Implementierung von komplexen, nebenläufigen Programmlogiken erleichtert.
Die Nutzung von Signalen und Slots erlaubt eine klare Trennung zwischen der interaktiven Benutzeroberfläche und der Hintergrundverarbeitung (Worker-Threads).
Worker-Threads können langwierige Aufgaben ausführen, ohne die Responsivität der GUI zu beeinträchtigen.
Wenn die Hintergrundverarbeitung abgeschlossen ist oder Zwischenstände erreicht sind, kann ein Signal ausgesendet werden, um die GUI zu aktualisieren. \cite{Qt52024}\\
Für den Benutzer steht in dieser Applikation ein Startfenster, bestehend aus einer Tabansicht mit Tabs für Single-Run und Multi-Run bereit.
In diesen Tabs können ein oder mehrere Programmdurchläufe mit festen Parametern (Gitterhöhe und -breite, Algorithmen, etc.) versehen werden.
Anschließend kann im Single-Run-Dialog das Lösen des Grids grafisch verfolgt oder im Multi-Run-Dialog ein Benchmark durchgeführt werden.

%********************************
\chapter{Eingesetztes Tooling}
\label{ch:tooling}
%********************************
In diesem Abschnitt soll auf die im Rahmen dieses Projekts verwendeten Tools eingegangen werden.
%********************************
\section{CMake}
\label{sec:tooling-cmake}
%********************************
CMake ist ein plattformübergreifendes Open-Source-Tool, das zur Steuerung des Kompiliervorgangs in Softwareprojekten verwendet wird.
Es ist insbesondere in größeren Projekten sehr hilfreich, da es die Projektkonfiguration und den Build-Prozess erheblich vereinfacht und standardisiert.
Anstatt separate Makefiles für unterschiedliche Betriebssysteme und Compiler zu schreiben, kann mit CMake eine einzige CMakeLists.txt-Datei erstellen, die unabhängig von der Zielplattform funktioniert und diese in spezifische Build-Skripte übersetzt. \cite{Cmake2024}\\\\
\textbf{Vorteile}
\begin{itemize}
    \item \textbf{Plattformunabhängigkeit}\\
    CMake ermöglicht die Erzeugung von Build-Skripten auf verschiedenen Plattformen wie Windows, macOS und Linux.
    Dies erleichtert die Pflege und Verteilung von Projekten, die auf mehreren Plattformen laufen müssen, da nur eine einzige Konfigurationsdatei benötigt wird.
    \item \textbf{Flexibilität}\\
    CMake unterstützt eine breite Palette von Compilern und Build-Systemen, einschließlich GNU Make, Ninja, MSBuild und Xcode.
    Dies gibt Entwicklern die Flexibilität, das für ihr Projekt am besten geeignete Build-System zu wählen, ohne die Build-Konfiguration anpassen zu müssen.
    \item \textbf{Modularität und Wiederverwendbarkeit}\\
    CMake erleichtert die Modularität, indem es dem Entwickler ermöglicht, Bibliotheken und Abhängigkeiten in separate CMakeLists.txt-Dateien aufzuteilen.
    Dies fördert die Wiederverwendbarkeit und erleichtert die Verwaltung von Abhängigkeiten, speziell in großen Projekten.
    \item \textbf{Automatische Abhängigkeitsverwaltung}\\
    CMake kann automatisch Abhängigkeiten zwischen den verschiedenen Komponenten eines Projekts erkennen und entsprechend steuern.
    Dies führt zu effizienteren Builds und einer einfacheren Fehlersuche, da der Build-Prozess genau steuert, welche Dateien neu kompiliert werden müssen.
\end{itemize}
\newpage
\textbf{Nachteile}
\begin{itemize}
    \item \textbf{Komplexität}\\
    Obwohl CMake viele Vorteile bietet, kann es aufgrund seiner Flexibilität, der Vielzahl an verfügbaren Optionen und der teilweise fehlenden Standardisierung kompliziert zu erlernen und zu verwenden sein.
    \item \textbf{Debugging-Optionen}\\
    Das Debugging von CMake-Skripten kann manchmal schwierig sein, insbesondere wenn Fehlkonfigurationen oder komplexe Skripte verwendet werden.
\end{itemize}

%********************************
\subsection{CTest und GoogleTest}
\label{subsec:ctest-integration}
%********************************
CMake bietet nicht nur umfangreiche Funktionen zur Konfiguration des Build-Prozesses, sondern unterstützt auch Testframeworks über das Modul \textit{CTest}.
CTest ist ein plattformübergreifendes Testtool, das speziell für die Integration in den CMake-Workflow konzipiert wurde.
Es ermöglicht das Erstellen, Ausführen und Verwalten von Tests in einer einfachen und konsistenten Weise, was die Qualitätssicherung und Wartung großer Softwareprojekte erheblich erleichtert.\\
In diesem Projekt wird das populäre Testframework \textit{GoogleTest} eingesetzt, welches eine flexible und leistungsfähige Lösung zur Erstellung von Unit- und Modultests bietet.
GoogleTest ist bekannt für seine leicht verständliche Syntax und umfangreiche Funktionalität. \cite{Cmake2024, GTest2024}
Im Rahmen dieses Projekts wurden folgende Testcases über dieses Framework integriert:
\begin{enumerate}
    \item \textbf{Unit-Tests}\\
    Für jeden eingesetzten Wegfindealgorithmus wurde ein eigenes Testszenario integriert.
    Hierfür wurde jeweils ein \texttt{Grid}-Objekt in eine binäre Datei serialisiert (über die Funktion \texttt{Grid::serialize()}).
    Dieses Objekt bildet ein mit dem jeweiligen Algorithmus gelöstes Gitter ab.
    Um die Funktionalität des Algorithmus zu testen, muss demnach nur die Binärdatei deserialisiert, zurückgesetzt, gelöst und anschließend mit dem Eingangszustand verglichen werden.
    \item \textbf{Headless Runner Module Test}\\
    Bei diesem Test handelt es sich um einen Integrationstest, bei dem das Gesamtmodul mittels \texttt{headless\_runner\_module\_test.cpp} getestet wird.
    Dieser Modultest soll die Funktionalität der Interfaces für kontinuierliches Lösen von Gittern sicherstellen.
    Das Testszenario ist hierbei durch vorgegebene Gitterparameter beschrieben, die über mehrere Iterationen von allen Algorithmen gelöst werden müssen.
\end{enumerate}
Durch die Einbindung und Verknüpfung der Tests mit der CMakeLists.txt wird gewährleistet, dass die Tests unmittelbar nach dem Build-Prozess ausgeführt werden können.
Dies steigert nicht nur die Zuverlässigkeit des Codes, sondern fördert auch den kontinuierlichen Verbesserungsprozess.
Im Folgenden ist die Integration eines beispielhaften Unit-Tests und des Modultests in \texttt{CMakeLists.txt} dargestellt:
\lstinputlisting[language=cmake, firstline=1, caption={CTest und GooleTest Integration},label={lst:tooling_ctest}]
{assets/listings/ctest.txt}

%********************************
\section{Docker DevContainer}
\label{sec:tooling-docker}
%********************************
\textit{DevContainer} basieren auf dem Prinzip der Containerisierung.
Containerisierung ist ein Konzept zur Virtualisierung auf Betriebssystemebene, das es ermöglicht, Anwendungen mit all ihren Abhängigkeiten und Konfigurationen in einem isolierten Paket namens „Container“ zu bündeln.
Dieser Container kann in jeder Umgebung ausgeführt werden – sei es auf einem lokalen Rechner, in der Cloud oder in einer CI/CD-Pipeline. \cite{Docker2024}\\\\
\textbf{Vorteile}
\begin{itemize}
    \item \textbf{Konsistente Entwicklungsumgebung}\\
    Alle Entwickler arbeiten in einer einheitlichen Umgebung, unabhängig vom Host-Betriebssystem.
    Dies verhindert das klassische Problem „it works on my machine“.
    \item \textbf{Isolierte Abhängigkeiten}\\
    Bibliotheken und Abhängigkeiten, die für das Projekt benötigt werden, sind im Container enthalten und beeinträchtigen nicht das Host-System.
    \item \textbf{Einfache Skalierbarkeit}\\
    Container können leicht repliziert und skaliert werden, was besonders in DevOps-Prozessen hilfreich ist.
\end{itemize}
Vor allem in Kombination mit CMake kann der Entwicklungsprozess über verteilte Maschinen so erheblich vereinfacht werden.
Die einheitliche Umgebung wir durch den Container bereitgestellt und der Buildprozess über CMake automatisiert, was den Prozess stabil und reproduzierbar macht.
\newpage
%********************************
\section{GitHub CI}
\label{sec:tooling-ci}
%********************************
\ac{CI} ist eine Softwareentwicklungspraxis, bei der Codeänderungen regelmäßig in das zentrale Repository integriert und automatisierte Builds und Tests durchgeführt werden.\\\\
\textbf{Vorteile}
\begin{itemize}
    \item \textbf{Frühzeitige Fehlererkennung}\\
    Automatisierte Tests und Builds bei jeder Codeänderung sorgen dafür, dass Fehler frühzeitig gefunden und behoben werden können.
    Dies reduziert in erster Linie den Aufwand für die Fehlerbehebung.
    \item \textbf{Qualitätssicherung}\\
    Durch kontinuierliche Tests wird die Codequalität aufrechterhalten und verbessert.
    CI-Tests helfen sicherzustellen, dass neue Änderungen die bestehende Funktionalität nicht beeinträchtigen.
    \item \textbf{Kontinuierliche Bereitstellung}\\
    CI ist oft der erste Schritt in Richtung Continuous Delivery (CD), bei dem die Software jederzeit in eine Produktionsumgebung bereitgestellt werden kann.
\end{itemize}
Durch die Integration von \textbf{Git}, \textbf{GitHub} und \textbf{CI} in das vorliegende Projekt wird ein schneller, sicherer und zentralisierter Entwicklungsprozess gewährleistet, der die Softwarequalität kontinuierlich überwacht und verbessert.

%\include{04_ergebnisse}

%\include{09_zusammenfassung_und_ausblick}

% ============= Literaturverzeichnis ============= %
\renewcommand{\baselinestretch}{1.2}\normalsize
\clearpage						% Zeilenabstand
\bibliographystyle{plaindin}
%\addcontentsline{toc}{chapter}{Literaturverzeichnis}
\bibliography{bibluS}
\renewcommand{\baselinestretch}{1}\normalsize						% Zeilenabstand

% =============Anhang ============= %
%\appendix
%%********************************
\chapter{Anhang}
\label{ch:appendix}
%********************************
\section{Beweis des Optimismus der euklidischen Distanz als Heuristik}
\label{sec:beweis_heuristik}
%********************************
\textbf{Beweis: Die Euklidische Distanz ist immer geringer als die Oktile Distanz und daher eine optimistische Heuristik.} \\
Zu zeigen:
\begin{equation}
    \label{eq:beweis_heuristik_bedingung}
    d_\text{eukl} \leq d_\text{okt}\)
\end{equation}

Wir betrachten die Distanzen für beliebige Punkte $(x_1, y_1)$ und $(x_2, y_2)$.\\
Betrachten wir den Fall, dass  $|x_2 - x_1| = |y_2 - y_1|$.
Dann gilt:
\begin{align*}
    d_\text{eukl} &= \sqrt{2} \cdot |x_2 - x_1|\\
    d_\text{okt} &= |x_2 - x_1| + (\sqrt{2}-1) \cdot |x_2 - x_1| = |x_2 - x_1| \cdot \sqrt{2}
\end{align*}
In diesem Fall sind die Distanzen identisch, was der Bedingung aus Gl. \ref{eq:beweis_heuristik_bedingung} entspricht.

Betrachten wir nun den Fall $|x_2 - x_1| \neq |y_2 - y_1|$.
Ohne Beschränkung der Allgemeinheit nehmen wir an $|x_2 - x_1| > |y_2 - y_1|$.
Dann gilt:
\begin{align*}
    d_\text{eukl} &= \sqrt{(x_2 - x_1)^2 + (y_2 - y_1)^2}\\
    d_\text{okt} &= |x_2 - x_1| + (\sqrt{2}-1) \cdot |y_2 - y_1|
\end{align*}

Nun muss die folgende Ungleichung bewiesen werden:
\begin{equation*}
    \sqrt{(x_2 - x_1)^2 + (y_2 - y_1)^2} \leq |x_2 - x_1| + |y_2 - y_1|
\end{equation*}

Da beide Seiten der Ungleichung nicht-negativ sind, ist das Quadrieren beider Seiten gültig und bewahrt die Richtung der Ungleichung.
Löst man den binomialen Term auf, ergibt sich:
\begin{equation*}
(x_2 - x_1)^2 + (y_2 - y_1)^2 \leq \left(|x_2 - x_1|\right)^2 + 2|x_2 - x_1||y_2 - y_1| + \left(|y_2 - y_1|\right)^2
\end{equation*}

Aufgrund von $|a|^2 = a^2$ für jede Zahl $a\in\mathbb{R}$ wird die Ungleichung zu:
\begin{equation*}
(x_2 - x_1)^2 + (y_2 - y_1)^2 \leq (x_2 - x_1)^2 + 2|x_2 - x_1||y_2 - y_1| + (y_2 - y_1)^2
\end{equation*}
Vereinfacht ergibt sich:
\begin{equation*}
    0 \leq 2|x_2 - x_1||y_2 - y_1|
\end{equation*}
Der Ausdruck $2|x_2 - x_1||y_2 - y_1|$ ist immer positiv, da es sich um das Produkt von Beträgen handelt, die selbst positiv sind.

Daraus folgt, dass für alle möglichen Paare $(x_1, y_1)$ und $(x_2, y_2)$:
\begin{equation*}
    d_\text{eukl} \leq d_\text{okt}
\end{equation*}
Damit ist gezeigt, dass die Euklidische Distanz eine optimistische Heuristik darstellt.

% ============= Leerseite ============= %
\newpage
\pagestyle{empty}
\phantom{t}


%%%%%%%%%%%%%%%%%%%%%%%%%%%%%%%%%%%%%%%%%%%%%%%%%%%%%%%%%%%%%%
% ============= 	  Ende des Dokuments  	   ============= %
%%%%%%%%%%%%%%%%%%%%%%%%%%%%%%%%%%%%%%%%%%%%%%%%%%%%%%%%%%%%%%
\end{document}
