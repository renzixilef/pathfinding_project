%\begin{longtable}{p{0.2\textwidth} p{0.8\textwidth}}
%    \hline
%    Symbol& Bedeutung\\
%    \hline
%    \endhead
%    $A_{ES}$            & Eshelby-Tensor\\
%    $A_{MT}$            & Einflusstensor des Mori-Tanaka-Modells\\
%    $\boldsymbol{C}$	& Steifigkeitsmatrix eines Systems\\
%    $c$                 & größte Übertragungsgeschwindigkeit eines Systems\\
%    $CFL$               & Courant-Friedrichs-Lewy-Zahl\\
%    $\boldsymbol{D}$	& Dämpfungsmatrix eines Systems\\
%    $d$                 & charakteristische Länge des RVE\\
%    $E$                 & Elastizitätsmodul\\
%    $\boldsymbol{f}$	& Lastvektor eines Systems\\
%    $f$	                & Fließfläche\\
%    $G$                 & Schubmodul\\
%    $\boldsymbol{I}$    & Einheitsmatrix\\
%    $K$                 & Kompressionsmodul\\
%    $k_i$               & Faktoren in \textit{MAT\_187}\\
%    $l$                 & charakteristische Länge der Inhomogenität\\
%    $L_{f}$         & Länge einer Faser\\
%    $L_{makro}$         & Ausdehnung der Makroebene\\
%    $\boldsymbol{M}$	& Massematrix eines Systems\\
%    $p$                 & mittlere Normalspannung\\
%    $R$                 & Festigkeit\\
%    $r_{f}$              & Faserradius\\
%    $\delta_{RVE}$      & Faktor auf die Größe des RVE\\
%    $\boldsymbol{S}$	& Nachgiebigkeitsmatrix eines Systems\\
%    $\Delta t$	        & diskreter Zeitschritt\\
%    $\boldsymbol{u}$	& Verschiebungsvektor eines Systems\\
%    $\boldsymbol{x}^{makro}$         & Ortsvektor eines RVE\\
%    $\Delta x_{min}$    & kleinster Abstand zwischen diskreten Knoten\\
%    $\gamma$            & Verzerrung\\
%    $\varepsilon$          & Dehnung\\
%    $\zeta$             & Faktor im Halpin-Tsai-Modell\\
%    $\eta$              & Faktor im Halpin-Tsai-Modell\\
%    $\nu$               & Poisson-Zahl/Querkontraktionszahl\\
%    $\theta$            & Faserwinkel zur Rotation um die y-Achse\\
%    $\nu_{\parallel\perp}$ & kleine Poisson-Zahl\\
%    $\nu_{\perp\parallel}$ & große Poisson-Zahl\\
%    $\sigma$            & Normalspannung\\
%    $\sigma_{VM}$       & von-Mises-Vergleichsspannung\\
%    $\tau$              & Schubspannung\\
%    $\phi$              & Faserwinkel zur Rotation um die z-Achse\\
%    $\varphi$           & Faservolumengehalt\\
%    \caption*{}
%\end{longtable}

\begin{table}[H]
    \begin{tabular}{ll}
        \hline
        Abkürzung\phantom{123456}& Bedeutung \phantom{1234567891234567891234567891234567899}\\
        \hline
    \end{tabular}
\end{table}
\begin{addmargin}[0.2cm]{0cm}
    \vspace{-0.6cm}
    \begin{acronym}[MMAE~~~~~~~~~~~~]
        \setlength{\itemsep}{-\parsep}
        \acro{JPS}{Jump-Point-Search}
        \acro{IDE}{Integrated Development Environment}
        \acro{CI}{Continuous Integration}
        \acro{GUI}{Graphical User Interface}
    \end{acronym}
\end{addmargin}