%********************************
\chapter{Programmaufbau und -struktur}
\label{ch:aufbau}
%********************************

%********************************
\section{Aufbau des Backends}
\label{sec:aufbau_backend}
%********************************
In diesem Abschnitt soll ein Überblick über den Aufbau und die Strukturierung des Backends gegeben werden.
`Backend' bezeichnet hierbei den ausführenden Teil des Programms, also Objekte und Strukturen die zur Berechnung/Lösung des Grids notwendig sind und keinen direkten User-Kontakt haben.

%********************************
\subsection{Überblick}
\label{subsec:aufbau_backend_ueberblick}
%********************************
Das Backend ist in zwei Namespaces unterteilt: `GridGenerator' und `Pathfinder'.
Ersterer beinhaltet jegliche Funktionalität die zur Erstellung der Grid-Objekte, zum Platzieren der Hindernisse und zur Positionierung des Start- und Endpunkts notwendig sind.
Der Namespace `Pathfinder' gruppiert Komponenten die zur Berechnung und Lösung ebendieser Objekte notwendig sind, sowie Funktionen zur Überwachung dieses Prozesses.
Im Folgenden soll zunächst auf die Erstellung der Grid-Objekte und sodann auf den Namespace `Pathfinder' eingegangen werden.
%********************************
\subsubsection{Namespace: GridGenerator}
\label{subsubsec:aufbau_backend_ueberblick_gridgen}
%********************************
Dieser Namespace setzt sich zusammen aus den Klassen \texttt{Cell}, \texttt{Grid} und \texttt{ObstacleGenerator}, sowie einigen Subklassen, unterstützenden Datenstrukturen und Enumerationen.
%\begin{itemize}
%    \item \textttt{Cell}
    \begin{figure}[H]
        \vspace{-0.5cm}
        \centering
        \includestandalone[width=0.7\textwidth]{assets/uml/cell_class_uml}
        \caption{UML-Klassendiagramm der Klasse \texttt{Cell}}
        \label{fig:uml_cell}
        %\vspace{-0.5cm}
    \end{figure}
%\end{itemize}