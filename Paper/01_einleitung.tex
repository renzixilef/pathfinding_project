%********************************
\chapter{Einleitung}
\label{ch:einleitung}
%********************************

%********************************
\section{Zielsetzung}
\label{sec:motivation}
%********************************
Im Rahmen des Moduls `Effizient Programmieren' im Masterstudiengang Luft- und Raumfahrttechnik an der Universität Stuttgart
wurde ein GUI-basiertes Programm zur Lösung von 2D Grids mithilfe diverser Pathfinding-Algorithmen in C++ entwickelt.
Ziel des Projekts war es, die effiziente Gestaltung von Codeprojekten und Entwicklungsprozessen zu erlernen und anzuwenden.
Der Schwerpunkt lag hierbei sowohl auf der algorithmischen Optimierung als auch auf der Implementierung moderner Softwareentwicklungsmethoden und -werkzeuge.

%********************************
\section{Projektbeschreibung}
\label{sec:beschreibung}
%********************************
Das entwickelte Programm ermöglicht die Visualisierung und Lösung von 2D Grids unter Verwendung verschiedener Pathfinding-Algorithmen wie A*, Dijkstra und \ac{JPS}.
Über ein \ac{GUI} können Anwender Start- und Zielpunkte definieren.
Die Algorithmen berechnen den optimalen Pfad und visualisieren diesen in Echtzeit.
Außerdem können im Multi-Run Modus des Programms laufzeitoptimierte Benchmarks der implementierten Algorithmen mit verschiedenen Grid-Konfigurationen durchgeführt werden.
%********************************
\subsection*{Verwendete Tools und Technologien}
%********************************
Zur Umsetzung des Projekts kamen unter anderem folgende Tools und Technologien zum Einsatz:
\begin{itemize}
\item \textbf{\textit{CMake}:} Einsatz als plattformunabhängiges Build-System zur effizienten Verwaltung von Abhängigkeiten und Compiler-Optionen.
\item \textbf{\ac{CI}:} Einrichtung einer CI-Pipeline zur kontinuierlichen Überprüfung des Codes und Ausführung automatisierter Tests.
\item \textbf{\ac{IDE}:} Verwendung von \textit{CLion} (JetBrains).
\item \textbf{Versionskontrolle:} Verwendung von \textit{Git} und \textit{GitHub} zur Versionskontrolle und zum nahtlosen Arbeiten auf verschiedenen Geräten.
\end{itemize}

%********************************
\subsection*{Optimierungen}
%********************************
Im Entwicklungsprozess wurden mehrere Optimierungstechniken angewendet, um die Effizienz des Programms zu steigern:
\begin{itemize}
\item \textbf{Algorithmische Optimierungen:} Verbesserung des A*- und \ac{JPS}-Algorithmus durch eine geeignete Wahl der Heuristikfunktion und den Einsatz effizienter Datenstrukturen.
\item \textbf{Speicheroptimierungen:} Reduktion des Speicherverbrauchs durch den Einsatz von smarten Zeigern und das Vermeiden unnötiger Kopien.
\item \textbf{Parallelisierung:} Parallelisierung von Berechnungsteilen und \ac{GUI}-Event-Loop zur Aufrechterhaltung der \ac{GUI}-Interaktion.
\end{itemize}

\subsection*{Lerneffekte}
Das Projekt ermöglichte wertvolle Erkenntnisse in verschiedenen Bereichen der effizienten Softwareentwicklung:
\begin{itemize}
\item Vertiefung moderner C++-Techniken und Best Practices.
\item Erfahrungen in der Einrichtung und Nutzung von CI-Pipelines zur Verbesserung der Codequalität und -stabilität.
\item Verbesserung der Fähigkeiten im Schreiben und Strukturieren von Tests.
\item Effiziente Nutzung von Entwicklungswerkzeugen und -umgebungen zur Steigerung der Produktivität.
\end{itemize}
Zusammenfassend hat das Projekt nicht nur zur Verbesserung der technischen Fähigkeiten beigetragen, sondern auch die Kompetenz in der Anwendung effizienter Entwicklungsprozesse und -methoden gestärkt.
In den folgenden Kapiteln werden die einzelnen Komponenten, Optimierungstechniken und Lerneffekte detailliert erläutert.



